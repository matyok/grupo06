\documentclass[a4paper,10pt]{article}
\usepackage[utf8]{inputenc}
\usepackage[brazil]{babel}
\usepackage{graphicx}
\usepackage{makeidx}
\usepackage{spverbatim}
\makeindex


%opening
\date{\today}
\begin{document}
	\begin{titlepage}
		\begin{figure}
			\centering
			\includegraphics[width=7cm,keepaspectratio=true]{imagens/unicamp.png}
	
		\end{figure}
		\begin{center}
			\huge{Universidade Estadual de Campinas}
		
		\vfill
		\textbf{\LARGE{Manual}}
		\vfill
		\end{center}
		
		\begin{flushleft}
			\begin{tabbing}
				Autores\qquad\qquad\= Gabriel Bueno de Oliveira 139455 \\
					\>Joao Guilherme Daros Fidelis 139715 \\
					\>Lucas Henrique Morais 136640 \\
					\>Matheus Yokoyama Figueiredo 137036\\
					\>Pedro Rodrigues Grijo 136242\\
			\end{tabbing}
		\end{flushleft}
	\end{titlepage}

\newpage
\section{Setup do Cluster}
    \subsection{Acesso}
    O acesso ao cluster pode ser feito de fora e de dentro do IC (Instituto de Computação - UNICAMP).
        \subsubsection{Interno ao IC}
        Para acceso interno ao ic, basta acessar por ssh o endereço cbn6.lab.ic.unicamp.br. No linux, isso pode ser feito através de:
        \begin{spverbatim}
        $ ssh cbn6.lab.ic.unicamp.br
        \end{spverbatim}
        
        Após a conexão ser feita, insira sua senha.
        \subsubsection{Externo ao IC}
        Para o acesso externo, é necessário fazer uma conexão de duas fases ou usar um tunnel por uma máquina do ic. Para acesso remoto a uma máquina do ic, no linux, use:
        \begin{spverbatim}
        $ ssh -l raXXXXXX ssh.students.ic.unicamp.br
        \end{spverbatim}
        
        Após a conexão ser feita, insira sua senha. Após feita a credencial, prossiga do mesmo modo como se tivesse acessando de dentro do IC.
        
\section{Instalação e Configuração}
    \subsection{PostgreSQL}
        \subsubsection{Instalação}
        A versão utilizada do PostgreSQL é a versão 9.5, que pode ser obtida seguindo os seguintes comandos:
        \begin{spverbatim}
        $ sudo su
        # cat /etc/apt/sources.list.d/postgresql.list
        # deb http://apt.postgresql.org/pub/repos/apt/ trusty-pgdg main
        # sudo apt-get install wget ca-certificates
        # wget --quiet -O - https://www.postgresql.org/media/keys/ACCC4CF8.asc | sudo apt-key add -
        # sudo apt-get update
        # sudo apt-get upgrade
        # sudo apt-get install postgresql-9.5
        
        \end{spverbatim}
        
        \subsubsection{Configuração}
        Para configurar o acesso ao banco de dados deve-se alterar o arquivo de configuração principal. Nele são definidos toda a configuração de acesso ao banco de dados e os diretórios de dados que são respectivamente: 
        
        \begin{spverbatim}
    hba_file = '/etc/postgresql/9.5/main/pg_hba.conf'
    data_directory = '/var/lib/postgresql/9.5/main'
        \end{spverbatim}
        
        O arquivo pg\_hba.conf define regras para a conexão ao banco. Por padrão, são só permitidas conexões locais: o usuário administrador do banco (postgres, quando instalado por gerenciador de pacotes) tem acesso via socket e sem senha. os demais usuários, via interface local e com senha:
        \begin{spverbatim}
        local   all         postgres            peer
    
        host    all         all         127.0.0.1/32            md5
        \end{spverbatim}
    
    Para criar o primeiro usuário siga os seguintes comandos:
        \begin{spverbatim}
            $ sudo su
            # su postgres
            # psql
            =# CREATE ROLE <usr_name> with superuser createdb createrole login;
            =# ALTER ROLE <usr_name> PASSWORD ‘<password>’;
        \end{spverbatim}
        
        \subsubsection{Carregando o Dump}
        Para carregar o ddump no banco deve-se criar o banco de dados (tpcw) e um usuário tpcw-user.
        \begin{spverbatim}
            $ pg_restore -d tpcw tpcw-database-dump.tar
        \end{spverbatim}
    
    
    \subsection{Tomcat}
        \subsubsection{Instalação}
        Para instalar o Tomcat (usaremos o tomcat7), precisamos instalar o pacote do tomcat e suas dependencias. Para isso, no linux, podemos usar o apt-get:
        \begin{spverbatim}
        $ sudo apt-get install tomcat7
        \end{spverbatim}
        
        A instalação irá criar um usuário tomcat7 dentro de um grupo, tomcat7. Os arquivos estão em \$CATALINA\_BASE, que na instalação do Ubuntu é /var/lib/tomcat7.

        Para teste, acesse em uma máquina do IC o cbn6.lab.ic.unicamp.br:8080.
        
        \subsubsection{Integração com o PostgreSQL}
        Para integração com o PostgreSQL é necessário fazer o download do .jar do PostgreSQL JDBC para ser adicionado a pasta \$CATALINA\_BASE/lib. Após fazer o download, modifique os seguintes arquivos com o conteúdo indicado:

        \begin{itemize}
        \setlength{\itemindent}{-.0in}
            \item \$CATALINA\_BASE/conf/context.xml: http://www.ggte.unicamp.br/eam/pluginfile.php/188108/
            mod\_wiki/attachments/132/tomcat7-context.xml
            \item \$CATALINA\_BASE/conf/server.xml: http://www.ggte.unicamp.br/eam/pluginfile.php/188108/
            mod\_wiki/attachments/132/tomcat7-server.xml. Para este, modifique os campos de username, password e possivelmente url.
        \end{itemize}
        
        Após essas configurações, reinicie o tomcat:
        
        \begin{spverbatim}
        # invoke-rc.d tomcat7 restart
        \end{spverbatim}
         
        \subsection{TPC-W}
        
        As aplicações implantadas no Tomcat são configuradas em \$CATALINA\_BASE/webapps. Para ter permissão de criar aplicações nesta pasta, o usuário tem que ser do grupo tomcat7, para fazer isso, no linux:
        
        \begin{spverbatim}
            # adduser [usuário_comum] tomcat7
        \end{spverbatim}
        
        Para instalação default do TPC-W no tomcat7, crie a seguinte árvore:
        \begin{spverbatim}
        
            $CATALINA_BASE/webapps
            
                tpcw
            
                     Images
            
                     WEB-INF
            
                        classes
            
                        web.xml
            
                        
        \end{spverbatim}
        
        Na pasta tpcw, rode os seguintes comandos:
        \begin{spverbatim}
            $ sudo cp /nfs/servlet/Images.tar.gz .
            $ sudo tar -xzf Images.tar.gz
            $ sudo cp /nfs/servlet/tpcw-servlets.tar.gz
            $ sudo tar -xzf tpcw-servlets.tar.gz
            $ cd tpcw-servlets.tar.gz
            $ sudo make
        \end{spverbatim}
        
        No fim, copie todas as *.class geradas para a pasta WEB-INF/classes e copie o web.xml para pasta WEB-INF/. Reinicie o Tomcat.
        
       \subsection{RBE}
        O RBE (Remote Browser Emulator) é o programa empregado para gerar carga para o TPC-W. É um aplicativo que emula um conjunto de clientes que acessam o lado servidor do TPC-W, que implementa uma loja de livros. Os clientes operam como se fossem navegadores web.
        O código fonte do RBE, assim como arquivo README e o script em Python que realiza a análise das saídas estão disponíves na pasta do RBE.
        
        \subsubsection{Automatização}
        Para facilitar a coleta dos dados foi implementado um script em bash que executa o rbe com variando os parâmetros () e as saídas, já executa o script analyse.py e gera os gráficos.
        Para executar o script basta executar:
        \begin{spverbatim}
            $ ./script.sh
        \end{spverbatim}
        
        
        
    
\end{document}
