% Exemplo de relatório técnico do IC
% Criado por P.J.de Rezende antes do Alvorecer da História.
% Modificado em 97-06-15 e 01-02-26 por J.Stolfi.
% Last edited on 2003-06-07 21:12:18 by stolfi
% modificado em 1o. de outubro de 2008
% modificado em 2012-09-25 para ajustar o pacote UTF8. Contribuicao de
%   Rogerio Cardoso

\documentclass[11pt,twoside]{article}
\usepackage{techrep-ic}
\usepackage{indentfirst}

%%% SE USAR INGLÊS, TROQUE AS ATIVAÇÕES DOS DOIS COMANDOS A SEGUIR:
\usepackage[brazil]{babel}
%% \usepackage[english]{babel}

%%% SE USAR CODIFICAÇÃO LATIN1, TROQUE AS ATIVAÇÕES DOS DOIS COMANDOS A
%%% SEGUIR:
%% \usepackage[latin1]{inputenc}
\usepackage[utf8]{inputenc}
\usepackage{graphicx}

\begin{document}

%%% PÁGINA DE CAPA %%%%%%%%%%%%%%%%%%%%%%%%%%%%%%%%%%%%%%%%%%%%%%%
% 
% Número do relatório
\TRNumber{02}

% DATA DE PUBLICAÇÃO (PARA A CAPA)
%
\TRYear{16}  % Dois dígitos apenas
\TRMonth{04} % Numérico, 01-12

% LISTA DE AUTORES PARA CAPA (sem afiliações).
\TRAuthor{Gabriel Oliveira \and Jo{\~a}o Fid{\'e}lis \and Lucas Morais \and Matheus Figueiredo \and Pedro Grij{\'o}}

% TÍTULO PARA A CAPA (use \\ para forçar quebras de linha).
\TRTitle{MC437 - Grupo06 - Relat{\'o}rio 2}

\TRMakeCover
%%%%%%%%%%%%%%%%%%%%%%%%%%%%%%%%%%%%%%%%%%%%%%%%%%%%%%%%%%%%%%%%%%%%%%
% O que segue é apenas uma sugestão - sinta-se à vontade para
% usar seu formato predileto, desde que as margens tenham pelo
% menos 25mm nos quatro lados, e o tamanho do fonte seja pelo menos
% 11pt. Certifique-se também de que o título e lista de autores
% estão reproduzidos na íntegra na página 1, a primeira depois da
% página de capa.
%%%%%%%%%%%%%%%%%%%%%%%%%%%%%%%%%%%%%%%%%%%%%%%%%%%%%%%%%%%%%%%%%%%%%%

%%%%%%%%%%%%%%%%%%%%%%%%%%%%%%%%%%%%%%%%%%%%%%%%%%%%%%%%%%%%%%%%%%%%%%
% Nomes de autores ABREVIADOS e titulo ABREVIADO,
% para cabeçalhos em cada página.
%
\markboth{Bueno, Fid{\'e}lis, Figueiredo, Grij{\'o}, Morais}{MC437 - Grupo06}
\pagestyle{myheadings}

%%%%%%%%%%%%%%%%%%%%%%%%%%%%%%%%%%%%%%%%%%%%%%%%%%%%%%%%%%%%%%%%%%%%%%
% TÍTULO e NOMES DOS AUTORES, completos, para a página 1.
% Use "\\" para quebrar linhas, "\and" para separar autores.
%
\title{MC437 - Grupo06}

\author{Gabriel Bueno de Oliveira \and
Jo{\~a}o Guilherme Daros Fid{\'e}lis \and
Lucas Henrique Morais \and
Matheus Yokoyama Figueiredo \and Pedro Rodrigues Grij{\'o}}
\date{}

\maketitle

%%%%%%%%%%%%%%%%%%%%%%%%%%%%%%%%%%%%%%%%%%%%%%%%%%%%%%%%%%%%%%%%%%%%%%

\begin{abstract} 
\setlength{\parindent}{4ex}

\end{abstract}

\section{Introdução}
\setlength{\parindent}{4ex}


\section{Condições Experimentais}
\setlength{\parindent}{4ex}
\begin{itshape}
	 Nesta se\c{c}\~ao ser\~ao descritas as configura\c{c}\~oes de hardware e software utilizadas nos experimentos, os par\^ametros dos RBEs e como foi feita a sincroniza\c{c}\~ao das m\'aquinas.
	
	Os experimentos foram feitos no cluster disponilizado pelo Instituto de Computa\c{c}\~ao(IC). A m\'aquina utilizada possui a seguinte configura\c{c}\~ao: sistema operacional Ubuntu 14.04, CPU Intel(R) Core(TM)2 Quad CPU 2.66GHz e mem\'oria RAM de 4GB e 1333 MHz.

	Utilizamos o servidor web Apache Tomcat vers\~ao 7 e PostgreSQL vers\~ao 9.5.1.

\end{itshape}

\section{Metodologia de Pesquisa}
\setlength{\parindent}{4ex}
\begin{itshape}
Para analisar a disponibilidade de nosso servidor, fizemos um script que executa o comando do RBE diversas vezes variando o n\'umero de usu\'arios para cada perfil. Por exemplo, Fizemos o RBE simular 100 usu\'arios com o perfil Shopping e medimos o WIPS m\'edio. Variamos o n\'umero de usu\'ario de 100 a 4000 incrementando esse n\'umero em 100 a cada vez. Tamb\'em, utilizamos os valores de \textit{Ramp-up Time} em 5s, \textit{Measurement Interval} em 30s, \textit{Ramp-down Time} em 5s e o resto dos par\^ametros no valor padr\~ao.
\end{itshape}

Valores utilizados - Ramp-Up Time: 5s, Ramp-Down Time: 5s e Measurement Interval: 90s

\section{An\'alise e Resultados}
\setlength{\parindent}{4ex}

Seguem os gráficos gerados pela execução do RBE para cada experimento

Experimento 1: sem replicação.
- Cargas: WIPSb (browsing), WIPS (shopping) e WIPSo (ordering).
Réplica primária ativa, réplica secundária desligada.

Experimento 2: sobrecarga gerada pela replicação.
- Cargas: WIPSb (browsing), WIPS (shopping) e WIPSo (ordering).
Réplica primária ativa, réplica secundária  ativa como hot standby.

Os  experimentos  1 e  2  devem  ser  utilizados para  caracterizar  o
desempenho  (operações/s  e tempo  de  resposta)  em função  da  carga
aplicada.

O  objetivo é  encontrar  a carga  que produz  o  melhor resultado  de
desempenho. Essa carga deve ser empregada no experimento 3.


Experimento 3: injeção de falha no primário.
- Cargas: WIPSb (browsing), WIPS (shopping) e WIPSo (ordering).
Replicado, com falha  do primário e promoção manual  do secundário.  Com
falha  injetada  aproximadamente  na  metade da  duração  do
experimento.

O objetivo  do experimento 3  é medir o período  em que o  sistema não
está disponível: quanto menor, melhor.  Realizem um número suficiente de
experimentos  que permita  que o  grupo  argumente que  o resultado  é
confiável  (reprodutível)  e  discuta   as  potenciais  causas  para  a
indisponibilidade.

- Think Time (TT)  deve ser otimizado  para  obtenção  de  melhor  desempenho.

- Banco  de  Dados: utilizar o dump fornecido.

Configuração para execução:

rbes em um ou mais nós do  bloco, tpc-w (tomcat) e haproxy no servidor
dedicado  a cada  grupo  e bancos  de dados  em  seus respectivos  nós
(dbmaster, dbslave ou dbmaster2, dbslave2).


\section{Conclus\~ao}

\end{document}
