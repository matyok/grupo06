% Exemplo de relatório técnico do IC
% Criado por P.J.de Rezende antes do Alvorecer da História.
% Modificado em 97-06-15 e 01-02-26 por J.Stolfi.
% Last edited on 2003-06-07 21:12:18 by stolfi
% modificado em 1o. de outubro de 2008
% modificado em 2012-09-25 para ajustar o pacote UTF8. Contribuicao de
%   Rogerio Cardoso

\documentclass[11pt,twoside]{article}
\usepackage{techrep-ic}

%%% SE USAR INGLÊS, TROQUE AS ATIVAÇÕES DOS DOIS COMANDOS A SEGUIR:
\usepackage[brazil]{babel}
%% \usepackage[english]{babel}

%%% SE USAR CODIFICAÇÃO LATIN1, TROQUE AS ATIVAÇÕES DOS DOIS COMANDOS A
%%% SEGUIR:
%% \usepackage[latin1]{inputenc}
\usepackage[utf8]{inputenc}

\begin{document}

%%% PÁGINA DE CAPA %%%%%%%%%%%%%%%%%%%%%%%%%%%%%%%%%%%%%%%%%%%%%%%
% 
% Número do relatório
\TRNumber{01}

% DATA DE PUBLICAÇÃO (PARA A CAPA)
%
\TRYear{16}  % Dois dígitos apenas
\TRMonth{03} % Numérico, 01-12

% LISTA DE AUTORES PARA CAPA (sem afiliações).
\TRAuthor{G. Oliveira \and J. Fid{\'e}lis \and M. Figueiredo \and P. Grij{\'o} \and L. Morais}

% TÍTULO PARA A CAPA (use \\ para forçar quebras de linha).
\TRTitle{MC437 - Grupo06}

\TRMakeCover

%%%%%%%%%%%%%%%%%%%%%%%%%%%%%%%%%%%%%%%%%%%%%%%%%%%%%%%%%%%%%%%%%%%%%%
% O que segue é apenas uma sugestão - sinta-se à vontade para
% usar seu formato predileto, desde que as margens tenham pelo
% menos 25mm nos quatro lados, e o tamanho do fonte seja pelo menos
% 11pt. Certifique-se também de que o título e lista de autores
% estão reproduzidos na íntegra na página 1, a primeira depois da
% página de capa.
%%%%%%%%%%%%%%%%%%%%%%%%%%%%%%%%%%%%%%%%%%%%%%%%%%%%%%%%%%%%%%%%%%%%%%

%%%%%%%%%%%%%%%%%%%%%%%%%%%%%%%%%%%%%%%%%%%%%%%%%%%%%%%%%%%%%%%%%%%%%%
% Nomes de autores ABREVIADOS e titulo ABREVIADO,
% para cabeçalhos em cada página.
%
\markboth{Bueno, Fid{\'e}lis, Figueiredo, Grij{\'o}, Morais}{MC437 - Grupo06}
\pagestyle{myheadings}

%%%%%%%%%%%%%%%%%%%%%%%%%%%%%%%%%%%%%%%%%%%%%%%%%%%%%%%%%%%%%%%%%%%%%%
% TÍTULO e NOMES DOS AUTORES, completos, para a página 1.
% Use "\\" para quebrar linhas, "\and" para separar autores.
%
\title{MC437 - Grupo06}

\author{Gabriel Bueno de Oliveira\thanks{} \and
Jo{\~a}o Guilherme Daros Fid{\'e}lis\thanks{} \and
Matheus Yokoyama Figueiredo\thanks{} \and Pedro Rodrigues Grij{\'o}\thanks{Minha m{\~a}e, meu pai, meu papagaio e a Xuxa.} \and Lucas Morais}

\date{}

\maketitle

%%%%%%%%%%%%%%%%%%%%%%%%%%%%%%%%%%%%%%%%%%%%%%%%%%%%%%%%%%%%%%%%%%%%%%

\begin{abstract} 
  Este trabalho é um relatório da primeira parte do projeto da displina, que consistiu em preparar a máquina remota, instalando o servidor Tomcat, o banco de dados PostgreSQL e integrando todas essas funcionalidades para fazer um site com dados gerados aleatoriamente. Também foi instalado na máquina o aplicativo TPC-W que é um benchmark de transações web. Para utilizar o TPC-W, foi instalado o RBE (Remote Browser Emulator), emula conjuntos de clientes que acessam o lado servidor do TPC-W, que implementa uma loja de livros. Os clientes operam como se fossem navegadores web.

  Com base nessas pesquisas, determinamos que ...
\end{abstract}

\section{Introdução}

 ...

\end{document}
