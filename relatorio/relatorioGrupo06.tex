% Exemplo de relat�rio t�cnico do IC
% Criado por P.J.de Rezende antes do Alvorecer da Hist�ria.
% Modificado em 97-06-15 e 01-02-26 por J.Stolfi.
% Last edited on 2003-06-07 21:12:18 by stolfi
% modificado em 1o. de outubro de 2008
% modificado em 2012-09-25 para ajustar o pacote UTF8. Contribuicao de
%   Rogerio Cardoso

\documentclass[11pt,twoside]{article}
\usepackage{techrep-ic}

%%% SE USAR INGL�S, TROQUE AS ATIVA��ES DOS DOIS COMANDOS A SEGUIR:
\usepackage[brazil]{babel}
%% \usepackage[english]{babel}

%%% SE USAR CODIFICA��O LATIN1, TROQUE AS ATIVA��ES DOS DOIS COMANDOS A
%%% SEGUIR:
%% \usepackage[latin1]{inputenc}
\usepackage[utf8]{inputenc}

\begin{document}

%%% P�GINA DE CAPA %%%%%%%%%%%%%%%%%%%%%%%%%%%%%%%%%%%%%%%%%%%%%%%
% 
% N�mero do relat�rio
\TRNumber{01}

% DATA DE PUBLICA��O (PARA A CAPA)
%
\TRYear{16}  % Dois d�gitos apenas
\TRMonth{03} % Num�rico, 01-12

% LISTA DE AUTORES PARA CAPA (sem afilia��es).
\TRAuthor{Gabriel Oliveira \and Jo{\~a}o Fid{\'e}lis \and Matheus Figueiredo \and Pedro Grij{\'o}}

% T�TULO PARA A CAPA (use \\ para for�ar quebras de linha).
\TRTitle{MC437 - Grupo06 - Relat{\'o}rio 1}

\TRMakeCover
%%%%%%%%%%%%%%%%%%%%%%%%%%%%%%%%%%%%%%%%%%%%%%%%%%%%%%%%%%%%%%%%%%%%%%
% O que segue � apenas uma sugest�o - sinta-se � vontade para
% usar seu formato predileto, desde que as margens tenham pelo
% menos 25mm nos quatro lados, e o tamanho do fonte seja pelo menos
% 11pt. Certifique-se tamb�m de que o t�tulo e lista de autores
% est�o reproduzidos na �ntegra na p�gina 1, a primeira depois da
% p�gina de capa.
%%%%%%%%%%%%%%%%%%%%%%%%%%%%%%%%%%%%%%%%%%%%%%%%%%%%%%%%%%%%%%%%%%%%%%

%%%%%%%%%%%%%%%%%%%%%%%%%%%%%%%%%%%%%%%%%%%%%%%%%%%%%%%%%%%%%%%%%%%%%%
% Nomes de autores ABREVIADOS e titulo ABREVIADO,
% para cabe�alhos em cada p�gina.
%
\markboth{Bueno, Fid{\'e}lis, Figueiredo, Grij{\'o}, Morais}{MC437 - Grupo06}
\pagestyle{myheadings}

%%%%%%%%%%%%%%%%%%%%%%%%%%%%%%%%%%%%%%%%%%%%%%%%%%%%%%%%%%%%%%%%%%%%%%
% T�TULO e NOMES DOS AUTORES, completos, para a p�gina 1.
% Use "\\" para quebrar linhas, "\and" para separar autores.
%
\title{MC437 - Grupo06}

\author{Gabriel Bueno de Oliveira \and
Jo{\~a}o Guilherme Daros Fid{\'e}lis \and
Matheus Yokoyama Figueiredo \and Pedro Rodrigues Grij{\'o}}
\date{}

\maketitle

%%%%%%%%%%%%%%%%%%%%%%%%%%%%%%%%%%%%%%%%%%%%%%%%%%%%%%%%%%%%%%%%%%%%%%

\begin{abstract} 
  Este trabalho \'e um relat\'orio da primeira parte do projeto da displina, que consistiu em preparar a m\'aquina remota, instalando o servidor Tomcat, o banco de dados PostgreSQL e integrando todas essas funcionalidades para fazer um site com dados gerados aleatoriamente. Tamb\'em foi instalado na m\'aquina o aplicativo TPC-W que \'e um benchmark de transa\c{c}\~oes web, que \'e uma aplica\c{c}\~ao Java. Para utilizar o TPC-W, foi instalado o RBE (Remote Browser Emulator), que emula conjuntos de clientes que acessam o lado servidor do TPC-W, que implementa uma loja de livros. O RBE \'e um simulador escrito completamente em Java que simula o tr\'afego HTTP que seria feito por um usu\'ario que estivesse acessando o site atrav\'es de um navegador.

  O TPC-W gera um n\'umero, o WIPS (Web Interactions per Second, n\'umero de itera\c{c}\~oes Web por segundo). O fluxo de trabalho \'e gerado pelo RBE e pode ser de tr\^es tipos diferentes de perfis. O perfil de compras (\textit{shopping}), onde 80\% das a\c{c}\~oes s\~ao de consulta e 20\% de escrita no banco de dados. O perfil de navega\c{c}\~ao (\textit{browsing}), tem 95\% das a\c{c}\~oes de leitura e 5\% de escrita. J\'a o perfil de compras (\textit{ordering}) tem metades de suas opera\c{c}\~oes de leitura e a outra metade de escrita.
\end{abstract}

\section{Introdu\c{c}\~ao}

 ...

\end{document}
